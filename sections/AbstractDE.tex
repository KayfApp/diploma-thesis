\section*{Kurzfassung}
\cfoot{Vennesa Belinić}

Robotik ist heute eine der wichtigsten Technologien im industriellen Umfeld. Durch die Kosteneffizienz und den Gesamtvorteil autonomer Robotersysteme sind diese und ihre Technologien sehr gefordert und stehen unter ständiger Entwicklung. Es ist der starke Einsatz und die Präsenz von Robotern, welche Unternehmen in Industrieländern von anderen mit kostengünstigeren Produktionsmöglichkeiten unterscheidet.\\
Mobile Robotersysteme werden meist für logistische Zwecke eingesetzt.
Dieses sind nicht immer in der Lage auf alle Änderungen ihrer Umgebung dynamisch zu reagieren. Dadurch können nicht nur Schäden an anderen Geräten zustande kommen, sondern auch Menschen die sich in deren Umgebung aufhalten in Gefahr gebracht werden.
Aufgrund dieser Einschränkung und auch wegen Sicherheitsbestimmungen können sich die meisten mobilen Robotersysteme nur auf vordefinierten Bahnen oder Arealen bewegen. Dies schränkt die Flexibilität des gesamten Systems ein.
Das Projekt \textit{RoboNav} verfolgt das Ziel, zu evaluieren ob eine flexiblere und genauere Navigation mit externen Bildverarbeitungssystemen aus Vogelperspektive erreicht werden kann.
\textit{RoboNav} hat die Systemumgebung so weit vereinfacht um dieses Szenario in diesem Schulprojekt nachzubauen. Trotzdem wird versucht noch den theoretischen industriellen Anforderungen zu entsprechen.
Das Proof-of-Concept umfasst vier Einheiten: das mobile Robotersystem \textit{ Robotino® v3} von Festo, die externe Bildaufnahme (\textit{RoboNav Sight}) mittels eins Android Smartphone mit einer Streaming-Applikation, eine Interaktionsmöglichkeit mit dem System durch eine intuitive graphische Oberflächen (\textit{RoboNav Overview}) und die Kontrollsoftware des Systems (\textit{RoboNav Control})  welche die wichtigsten Aspekte dieses Projektes beinhaltet. \textit{RoboNav Control} verhält sich wie eine Middleware und verbindet alle anderen Komponenten miteinander, teils durch ein selbst definiertes Kommunikationsprotokoll (\textit{RNCP}). Die Kontrollsoftware beinhaltet auch die Bildbeschaffung und -verabeitung und stellt Pfadplanungs- und Pfadbefahrungsfunktionalitäten zu Verfügung. Durch den Abgleich der tatsächlichen Position mit der internen vom Roboter berechneten ist der wichtigste Part des Projekts abgedeckt.
Das Projekt ist als Machbarkeitsstudie zu betrachten, trotzdem sollte das Resultat eine Software sein die sich ohne große Designänderungen in komplexere, andere Systeme einbauen lässt.
